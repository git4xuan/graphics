\documentclass[UTF8]{article}
\usepackage{ctex}
\usepackage{listings}
\usepackage[bf,small,indentafter,pagestyles]{titlesec}
\usepackage{url}
\begin{document}
%\titleformat{\chapter}[hang]{\centering\LARGE\bfseries}{\chaptername}{1em}{}
\author{彭峰  05121214}
\date{2015-4-7}
\title{ 计算机图形学第一次作业  } 
\titlespacing{\chapter}{0pt}{*0}{*4}
\titlelabel{\S\thetitle\quad}

\tableofcontents
~
~
\maketitle
%\chapter{章节}没有渲染,可能少包了。
\section{前言}
这是任选计算机图形学的第一次编程作业,要求给出简单解题思路、计算结果、结论、程序代码。
相关的图片部分放于网上,作业放于github上。
\section{相关解答}
\subsection{前置知识了解}
这次作业查阅了一定量资料,由于未知的原因CNKI一直登不上(真是见鬼,那么晚都没登上),所以资料主要来源于Google。下面简单写写。
\subsubsection{CIE xy色度图}
因为人类眼睛有响应不同波长范围的三种类型的颜色传感器,所有可视颜色的完整绘图是三维的。但是颜色的概念可以分为两部分:明度和色度。例如,白色是明亮的颜色,而灰色被认为是不太亮的白色。换句话说,白色和灰色的色度是一样的,而明度不同。
CIE XYZ色彩空间故意设计得Y参数是颜色的明度或亮度的测量。颜色的色度接着通过两个导出参数x和y来指定,它们是所有三个三色刺激值X、Y和Z的函数所规范化的三个值中的两个:
\begin{equation} \label{eq:solve}
x = \frac{X}{X+Y+Z}
\end{equation}
\begin{equation}
y = \frac{Y}{X+Y+Z}
\end{equation}
\begin{equation}
z = \frac{Z}{X+Y+Z} = 1 - x - y
\end{equation}
导出的色彩空间用x, y, Y来指定,它叫做CIE xyY色彩空间并在实践中广泛用于指定颜色。
X和Z三色刺激值可以从色度值x和y与Y三色刺激值计算回来:
\begin{equation}
X=\frac{Y}{y}x
\end{equation}
\begin{equation}
Z=\frac{Y}{y}(1-x-y)
\end{equation}

\subsubsection{色品坐标}
CIE三原色光
红原色光    R=700.00nm
绿原色光    G=546.1nm
蓝原色光    B=435.8nm
色品坐标


三原色各自在R+G+B总量中的相对比例(记为r, g, b)
CIE RGB空间可以被用来以常规方式定义色度:色度坐标是r和g:


\begin{equation}
r= \frac{R}{R+G+B}
\end{equation}

\begin{equation}
g= \frac{G}{R+G+B}
\end{equation}

\subsubsection{判断坐标是否在要求的色域三角形内}
主要是判断坐标是否在色域三角形内,转为判断一个点是否在指定的三角形内
。算法的方法较多,主要有下面几种:
\begin{enumerate}
\item  通过计算三个小三角形的面积是否和三角形的面积相等来判断
\item  通过向量叉乘的方法判断
\item  通过向量点乘的方法判断
\end{enumerate}

这里不做过多叙述。

\subsection{计算结果}

\lstset{numbers=none,
  numberstyle=\scriptsize,
  frame=none,
  flexiblecolumns=false,
  language=,
  basicstyle=\ttfamily\tiny,
  breaklines=true,
  extendedchars=true,
  escapechar=~,
  texcl=true,
  showstringspaces=false,
  keywordstyle=\bfseries,
  xleftmargin=0em,
  xrightmargin=-6em,
  aboveskip=1em,
  tabsize=4}

\begin{lstlisting}[language=C, label=lst:4coloroutput, caption=Answers, numbers=left, stepnumber=1, frame=leftline , texcl=true, basicstyle=\ttfamily]

YOU SHOULD INPUT COLOR_X  , COLOR_Y  , COLOR_Z

THE OUTPUT IS:

97.98 100.00 95.75
#######################Color1########################
##                           Status                ##
##  Device:Galaxy               1                  ##
##  Device:iPad2                1                  ##
##  Standard:AdobeRGB           1                  ##
##  Standard:sRGB               1                  ##
##                                                 ##
## When Status is 1  , it means This color can be  ##
## displayed in Devices/Standard ,if Status is 0 , ##
## it means can not be.                            ##
#####################################################



if you want to test more Colors' CIE-XYZ ,Please input
Color_X  Color_Y Color_Z
47.35 85.00 12.40
#######################Color2########################
##                           Status                ##
##  Device:Galaxy               1                  ##
##  Device:iPad2                1                  ##
##  Standard:AdobeRGB           1                  ##
##  Standard:sRGB               0                  ##
##                                                 ##
## When Status is 1  , it means This color can be  ##
## displayed in Devices/Standard ,if Status is 0 , ##
## it means can not be.                            ##
#####################################################



if you want to test more Colors' CIE-XYZ ,Please input
Color_X  Color_Y Color_Z
131.72   211.00  344.70
#######################Color3########################
##                           Status                ##
##  Device:Galaxy               1                  ##
##  Device:iPad2                0                  ##
##  Standard:AdobeRGB           1                  ##
##  Standard:sRGB               0                  ##
##                                                 ##
## When Status is 1  , it means This color can be  ##
## displayed in Devices/Standard ,if Status is 0 , ##
## it means can not be.                            ##
#####################################################



if you want to test more Colors' CIE-XYZ ,Please input
Color_X  Color_Y Color_Z
19.18    72.00   11.68
#######################Color4########################
##                           Status                ##
##  Device:Galaxy               1                  ##
##  Device:iPad2                0                  ##
##  Standard:AdobeRGB           0                  ##
##  Standard:sRGB               0                  ##
##                                                 ##
## When Status is 1  , it means This color can be  ##
## displayed in Devices/Standard ,if Status is 0 , ##
## it means can not be.                            ##
#####################################################



if you want to test more Colors' CIE-XYZ ,Please input
Color_X  Color_Y Color_Z


\end{lstlisting}


\subsection{结论}
\subsubsection{两种设备的比较}
\begin{itemize}
\item  这里只对色域部分进行简单比较,多点测试时表明Galaxy的可能更广
\item  Galaxy 四点测试全部通过,iPad通过两点。
\end{itemize}
\subsubsection{两种标准的比较}
\begin{itemize}
\item  已知AdobeRGB的色域比sRGB的色域更广
\item  AdobeRGB通过三点,sRGB通过两点
\end{itemize}
\subsubsection{其他}
\begin{itemize}
\item  本次由于没有使用MATLAB作图,所以无法判断色域包含情况和色域位置。
\item  无法详细比较色域的重叠情况,无法详细输出设备覆盖AdobeRGB和sRGB的范围。
\item  好像Galaxy屏幕比较牛逼,当时是AMOLED屏?接着iPad2还在IPS那里混?
\end{itemize}


\subsection{程序代码}
%%%%%%%%%%%%%%%%应该是这里的章节将下面的代码包住了,导致了不能正常的输出,需要将这个部分包起来。这是部分不兼容的原因,还是不要使用CJK的好,GBK编码太坑爹,GB10830这个扩展的编码方式简直没救了。

\subsubsection{源代码}

\lstset{numbers=none,
  numberstyle=\scriptsize,
  frame=none,
  flexiblecolumns=false,
  language=,
  basicstyle=\ttfamily\tiny,
  breaklines=true,
  extendedchars=true,
  escapechar=~,
  texcl=true,
  showstringspaces=false,
  keywordstyle=\bfseries,
  xleftmargin=0em,
  xrightmargin=-6em,
  aboveskip=1em,
  tabsize=4}

\begin{lstlisting}[language=C, label=lst:Sources, caption=Sources, numbers=left, stepnumber=1, frame=leftline , texcl=true, basicstyle=\ttfamily]

#include <stdio.h>
#include <stdlib.h>
#include <math.h>

#define ABS_FLOAT_0  0.00001


int main()
{
    typedef struct point{
        float x;
        float y;
    }point;
    typedef struct triangle{
        point A;
        point B;
        point C;
    }triangle;

    float Side_length (point M , point N ){
        float length ;
        length = fabs( sqrt((M.x - N.x)*(M.x - N.x) + (M.y - N.y)*(M.y - N.y)) );
        return length ;
    }

    int Is_triangle(triangle one){  //判断


        float AB ,AC ,BC ;
        AB = Side_length(one.A , one.B );
        AC = Side_length(one.A , one.C );
        BC = Side_length(one.B , one.C );

        if ( ( (AB*AC*BC)> 0 )&&( (AB+BC) > AC )&&( (AB+AC) > BC ) &&( (AC+BC) > AB )  )
            return 1;
        else
            return 0;
    }
    float Get_Triangle_Square (point A , point B ,point C ){
        float AB ,AC ,BC ;
        float  square , semi_perimeter ;
        AB = Side_length(A , B );
        AC = Side_length(A , C );
        BC = Side_length(B , C );
        semi_perimeter =( AB + AC + BC )/2 ;
        square = sqrt(semi_perimeter * (semi_perimeter - AB ) * (semi_perimeter - BC ) * (semi_perimeter - AC ) );
        return square;
    }

    int Is_display(point CIE_point , point A , point B ,point C){
        float SABC , SADB , SADC , SBDC ;
        float SumSquare ;
        SABC = Get_Triangle_Square(A ,B ,C);
        SADB = Get_Triangle_Square(A, CIE_point ,B);
        SADC = Get_Triangle_Square(A , CIE_point , C);
        SBDC = Get_Triangle_Square(B ,CIE_point , C );
        SumSquare = SADC + SADB + SBDC ;
        if((-ABS_FLOAT_0 < ( SABC - SumSquare )) && ((SABC - SumSquare) < ABS_FLOAT_0   )){
            return 1;
        }
        else{
        return 0 ;
        }
    }

    float color_x , color_y , color_z ;
    float CIE_x , CIE_y;
    point CIE_point;
    triangle Galaxy , iPad , AdobeRGB , sRGB ;
    int status_G = -1 , status_i = -1 , status_A = -1 , status_s = -1 ;
    int count = 0;

       (Galaxy.A).x = 0.6627 ; (Galaxy.A).y = 0.3365  ;
       (Galaxy.B).x = 0.1750 ; (Galaxy.B).y = 0.7315  ;
       (Galaxy.C).x = 0.1440 ; (Galaxy.C).y = 0.0431  ;
       (iPad.A).x   = 0.6476 ; (iPad.A).y   = 0.3293  ;
       (iPad.B).x   = 0.3062 ; (iPad.B).y   = 0.6109  ;
       (iPad.C).x   = 0.1525 ; (iPad.C).y   = 0.0454  ;
       (AdobeRGB.A).x = 0.64; (AdobeRGB.A).y = 0.33 ;
       (AdobeRGB.B).x = 0.21; (AdobeRGB.B).y = 0.71 ;
       (AdobeRGB.C).x = 0.15; (AdobeRGB.C).y = 0.16 ;
       (sRGB.A).x   = 0.64; (sRGB.A).y = 0.33 ;
       (sRGB.B).x   = 0.30; (sRGB.B).y = 0.60 ;
       (sRGB.C).x   = 0.15; (sRGB.C).y = 0.06 ;



    while (scanf(" %f %f %f " , &color_x,&color_y,&color_z) != EOF ){

       CIE_x = color_x/(color_x + color_y + color_z);
       CIE_y = color_y/(color_x + color_y + color_z);

       CIE_point.y = CIE_y;

       status_G = -1 , status_i = -1 , status_A = -1 , status_s = -1 ;
       status_G = Is_display (CIE_point , Galaxy.A , Galaxy.B , Galaxy.C );
       status_i = Is_display (CIE_point , iPad.A , iPad.B , iPad.C );
       status_A = Is_display (CIE_point , AdobeRGB.A , AdobeRGB.B , AdobeRGB.C );
       status_s = Is_display (CIE_point , sRGB.A , sRGB.B , sRGB.C );

       count++;
       printf("#######################Color%d########################\n",count );
       printf("##                           Status                ##\n");
       printf("##  Device:Galaxy               %d                  ##\n",status_G );
       printf("##  Device:iPad2                %d                  ##\n",status_i );
       printf("##  Standard:AdobeRGB           %d                  ##\n",status_A );
       printf("##  Standard:sRGB               %d                  ##\n",status_s );
       printf("##                                                 ##\n");
       printf("## When Status is 1  , it means This color can be  ##\n");
       printf("## displayed in Devices/Standard ,if Status is 0 , ##\n");
       printf("## it means can not be.                            ##\n");
       printf("#####################################################\n");
       printf("\n");
       printf("\n");
       printf("\n");
       printf("if you want to test more Colors' CIE-XYZ ,Please input \n");
       printf("Color_X  Color_Y Color_Z  \n");
    }
    return 0;
}
\end{lstlisting}            
~
\subsubsection{简单解释}

\begin{lstlisting}[language=,texcl=true, basicstyle=\ttfamily]

判断三角形存在函数 Is_triangle(),不过这次只是写上去,并没有使用。
求三角形面积函数 Get_Triangle_Square()。
判断要求的点是否在三角形内的函数Is_display(),使用三角形面积判断的方式。两种设备和两个标准的相关值被固化在程序内部。
输出没有更人性化,直接将返回值返回了。

\end{lstlisting}
\subsubsection{缺陷}

\begin{enumerate}
\item   扩展性不好,只是习惯性的可以输入多个数值。没有可以输入多个色域的函数。
\item   输出没有更人性化,没有提示输入,输出只使用了0 和 1 表示。
\item   偷懒没有使用MATLAB,额,其实已经好久没有用过了,还占地方。
\end{enumerate}

\subsection{参考资料}
二维平面上判断点是否在三角形内~

\url{http://www.cnblogs.com/TenosDoIt/p/4024413.html}

CIE的Presentation~

\url{https://www.academia.edu/8313696/第二章_CIE标准色度系统1}

WiKi~
\url{http://zh.wikipedia.org/wiki/CIE1931色彩空间}

题目的图片~

\url{http://fengidea.qiniudn.com/graphics-20150407-002.jpg}

\end{document}