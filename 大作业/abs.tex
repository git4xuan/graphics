\documentclass[UTF8]{article}
\usepackage{ctex}
\usepackage{listings}
\usepackage[bf,small,indentafter,pagestyles]{titlesec} 

\usepackage{geometry}  %边距和纸张大小
\geometry{a4paper}
\geometry{left=2cm,right=2cm,top=3cm,bottom=3cm}
\usepackage{fancyhdr}
\usepackage{url}
 %使用graphicx包
\usepackage{graphicx}
\usepackage{amsmath}
\usepackage{amssymb} %使用数学符号使用 
\usepackage{cite} %使用引用 
\usepackage{enumerate} %使用item使用

\begin{document} 
\author{    彭峰     05121214   }
 \date{2015-5-31}
\title{计算机图形学}  %题目要改掉   后面
 \titlespacing{\chapter}{0pt}{*0}{*4}
\titlelabel{\S\thetitle\quad}
\maketitle%这是无目录只有title的部分


\section{前言}
计算机图形学(Computer Graphics,简称CG)是一种使用数学算法将二维或三维图形转化为计算机显示器的栅格形式的科学。简单地说,计算机图形学的主要研究内容就是研究如何在计算机中表示图形、以及利用计算机进行图形的计算、处理和显示的相关原理与算法。图形通常由点、线、面、体等几何元素和灰度、色彩、线型、线宽等非几何属性组成。从处理技术上来看,图形主要分为两类,一类是基于线条信息表示的,如工程图、等高线地图、曲面的线框图等,另一类是明暗图,也就是通常所说的真实感图形。经过30多年的发展,计算机图形学已成为计算机科学中最为活跃的分支之一,并得到广泛的应用。其中,3D 计算机图形技术作为计算机图形学理论中近期最为火热的一项技术,在影视特效中的运用已经远远超越了“电脑特技”的概念。它不但可以使用虚拟技术模式现实场景,极大地减少拍摄成本,更重要的是它的表现力可以超越现实事物,甚至完美的刻画现实中无法达到的画面和效果。
在这里,本文分别从常见的硬件设备,图像数字化等等方面进行阐述,旨在了解计算机图形学可能使用的主要知识点。此外我们假设一些常见的名词是已知的,并不对其作出解释。

\section{研究内容}
如何在计算机中表示图形,以及如何利用计算机进行图形的生成、处理和显示的相关原理与算法,构成了计算机图形学的主要研究内容。从处理技术上来看,图形主要分为两类,一类是由线条组成的图形,如工程图、等高线地图、曲面的线框图等,另一类是类似于照片的明暗图(Shading),也就是通常所说的真实感图形。

可以说,计算机图形学的一个重要研究内容就是要利用计算机产生令人赏心悦目的真实感图形。计算机图形学与另一门学科-计算机辅助几何设计有着密切的关系。事实上,图形学也把可以表示几何场景的曲线曲面造型技术和实体造型技术作为其重要的研究内容。同时,真实感图形计算的结果是以数字图象的方式提供的,计算机图形学也就和图象处理有着密切的关系。图形与图象两个概念间的区别越来越模糊,但我们认为还是有区别的:图象纯指计算机内以位图(Bitmap)形式存在的灰度信息。

计算机图形学的研究内容非常广泛,如图形硬件、图形标准、图形交互技术、光栅图形生成算法、曲线曲面造型、实体造型、真实感图形计算与显示算法,以及科学计算可视化、计算机动画、自然景物仿真、虚拟现实等。

\section{硬件设备}
硬件设备主要是将自然界的图像转变为计算机能够识别的图像,这是一个输入输出转化的过程,在进行图像记录和输入时,一些硬件是需要的,同时,为了能够将数字图像以某种形式输出,我们需要一些输出设备。
图像数字化需要有下面五个部分:
采样孔:用于减小周围噪声,定点观测特定的图像元素,可以添加过滤装置。
图像扫描结构:设备的框架部分,用来良好的收集到图样。
光传感器:一般是将光强参数转变为电压或者电流的变换器。
量化器:函数表达方式通常用连续函数和离散函数,这里通常是将连续的物理量转变为离散的物理量,一般情况是模数转换电路。
存储:输出的大量数值存储到存储器中便于进行后期的处理和加工。
常见的设备有:扫描仪,摄像机,照相机等。


\subsection{扫描仪}
是一个能够把照片、印刷文件或手写文件等视频,或装饰品等小对象扫描、分析并化成数字视频的器材。通常可通过感光CCD 阅读红绿蓝3原色(RGB) 数据。存在时间较长,通常只在部分专业领域使用。普通的可以通过手机来代替。
\subsection{摄像机}
此处的图像传感器包括光电二极管阵列,电荷耦合器件,CMOS传感器,光电导成像四种的转换方式。
光电二极管通过包含一个光电二极管传感器阵列,一组开关和相关控制电路的方式,来实现光电转换。
电荷耦合器件:是一个基于模拟信号的设备。当光投射到其表面时,将有信号电荷产生。电荷信号可以被转换成电压,并按指定的时序将图像信息输出。数码相机主板上的其他电路将把这信号转换成数字信号,以便微处理器进行处理。
CMOS传感器器件:每个光电传感器附近都有相应的电路直接将光能量转换成电压信号。功耗较低。
光电导成像:其类似于CRT的扫描和成像方式。光学图像转换后通过电子打击局部放电的形式形成电子图像。
现主要使用的是CCD和CMOS传感器,其他已经接近淘汰。
\subsection{照相机}
照相机是任何可以捕捉和记录影像的设备。最常见的照相机拍摄可见光的图像,但并不是所有照相机都需要可见光(如红外线热像仪),有的甚至不需要一个传统意义上的光源(如扫描隧道显微镜)。很多设备都具备照相机的特征,如雷达、医学成像设备、天文观测设备等等。
\subsection{显示器}
主要的包括CRT显示器和液晶显示器,CRT显示器已经淘汰,液晶显示器经过多代的更新后已是市场主流,有多种不同的类型。常见的有LCD,LED,3D显示技术。
\subsection{打印机}
可以将电脑内储存的数据按照文字或图形的方式永久的输出到纸张、透明胶片或其他平面媒介上。常见的有:点阵打印机,激光打印机,喷墨打印机,热传导打印机。
\subsection{计算机}
重要的处理平台,计算时使用CPU或者GPU进行计算。
\subsection{其他}
除此之外还有人们所熟悉的键盘,鼠标,还有图形输入板、扫描仪、光笔、游戏杆、跟踪球、触摸屏和语音系统等。


\section{常见API,平台和标准}
\subsection{PostScript}
PostScript(PS)是主要用于电子产业和桌面出版领域的一种页面描述语言和编程语言。PostScript是一种基于堆栈的解释语言(例如stack language),它类似于Forth语言但是使用从Lisp语言派生出的数据结构。这种语言的语法使用逆波兰表示法,这就意味着不需要括号进行分区,但是因为需要记住堆栈结构,所以需要进行训练才能阅读这种程序。大部分运算符(其他程序中称为函数)从堆栈中读取变量,并且将运算结构放到堆栈中。
\subsection{OpenGL}
开放图形库(英语:Open Graphics Library,缩写为OpenGL)是个定义了一个跨编程语言、跨平台的应用程序接口(API)的规范,它用于生成二维、三维图像。这个接口由近三百五十个不同的函数调用组成,用来从简单的图形比特绘制复杂的三维景象。而另一种程序接口系统是仅用于Microsoft Windows上的Direct3D。OpenGL常用于CAD、虚拟实境、科学可视化程序和电子游戏开发。

OpenGL的高效实现(利用了图形加速硬件)存在于Windows,很多UNIX平台和Mac OS。这些实现一般由显示设备厂商提供,而且非常依赖于该厂商提供的硬件。
\subsection{OpenCV}
OpenCV的全称是Open Source Computer Vision Library,是一个跨平台的计算机视觉库。OpenCV是由英特尔公司发起并参与开发,以BSD许可证授权发行,可以在商业和研究领域中免费使用。OpenCV可用于开发实时的图像处理、计算机视觉以及模式识别程序。OpenCV用C++语言编写,它的主要接口也是C++语言,但是依然保留了大量的C语言接口。该库也有大量的Python, Java and MATLAB/OCTAVE (版本2.5)的接口。这些语言的API接口函数可以通过在线文档获得。[4]现在也提供对于C\#, Ruby的支持。所有新的开发和算法都是用C++接口。一个使用CUDA的GPU接口也于2010年9月开始实现.
\subsection{DirectX}
DirectX(Direct eXtension,缩写:DX)是由微软公司创建的一系列专为多媒体以及游戏开发的应用程序接口。旗下包含了Direct3D、Direct2D、DirectCompute等等多个不同用途的子部分,因为这一系列API皆以Direct字样开头,所以DirectX(只要把X字母替换为任何一个特定API的名字)就成为了这一巨大的API系列的统称。目前最新版本为DirectX 12,随附于Windows 10操作系统之上。

\subsection{Qt}
Qt是一个跨平台的C++应用程序开发框架。广泛用于开发GUI程序,这种情况下又被称为部件工具箱。也可用于开发非GUI程序,比如控制台工具和服务器。Qt使用标准的C++和特殊的代码生成扩展(称为元对象编译器(Meta Object Compiler, moc))以及一些宏。
\subsection{GTK+}
GTK+最初是GIMP的专用开发库(GIMP Toolkit),后来发展为Unix-like系统下开发图形界面的应用程序的主流开发工具之一。GTK+是自由软件,并且是GNU计划的一部分。GTK+的许可协议是LGPL。GTK+使用C语言开发,但是其设计者使用面向对象技术。也提供了C++(gtkmm)、Perl、Ruby、Java和Python(PyGTK)绑定,其他的绑定有Ada、D、Haskell、PHP和所有的.NET编程语言。


%% \section{常用处理软件}  同上
%\section{基本图形计算方法}
%\subsection{点运算}
%\subsection{代数运算}
%\subsection{几何运算}
%%这里的有些问题,需要准备好



\section{图像处理}
\subsection{ 图像复原}
图像复原,我们知道,图像的获取是会有噪声存在的,包括在光学系统、运动等造成的图像模糊,以及在电路中和光学因素,都会产生噪声。对于此类退化的图像,我们可以进行处理使其趋向于复原成没有退化的理想图像。
其通常有两种方法,一种是在已知相应的退化模型的情况下,重建或恢复原始的图像。另一种在无法构建已知模型的情况下,将图片blabla。
常用的方法,有滤波技术,小波变换,空间复原,模型等。
\subsection{图像压缩}

有损压缩和无损压缩, ..冗余的方式等。
压缩的技术方式主要有两种,有损压缩和无损压缩,无损压缩主要是在文档的保存上很有用处,有损压缩在对视频
和图像上使用较多,这些应用中,出现一定量的错误是可以接受的。

冗余的存在是由于数据的重复和不必要的数据的存在。冗余是可以被量化的。
冗余的主要方式是有三种:编码冗余、像素间冗余、心理视觉冗余。
\subsection{编码冗余}
编码冗余:是一种很常见的冗余方式,通过使用变长编码方式可以实现数据压缩。
像素间冗余:我们举个示例,两张不同的灰度图,当两幅图像有着相同的直方图时。各种灰度值的出现概率不是等可能性的,可以通过使用变长编码减少对像素进行统一长度的编码和自然二进制编码带来的冗余。
心理视觉冗余:人类感知图像并不是针对像素点的定量分析,而是通过可识别的特征构建一些组群,和之前的一些知识联系,构建感知图像。

%\section{模式识别}
%%这里看的自己会不会写,反正。。。
\section{计算机图形学应用}
几乎各行各业都多少要用到计算机图像学,但主要涉及以下行业:
视频游戏、动画片、电影特效、CAD/CAM、仿真、医学成像、信息可视化。 计算机图形学是随着计算机及其外围设备而产生和发展起来的,作为计算机科学与技术学科的一个独立分支已经历了近40年的发展历程。一方面,作为一个学科,计算机图形学在图形基础算法、图形软件与图形硬件三方面取得了长足的进步,成为当代几乎所有科学和工程技术领域用来加强信息理解和传递的技术和工具。。

\subsection{计算机辅助设计与制造}

这是一个最广泛,最活跃的应用领域。计算机辅助设计(Computer Aided Design,CAD)是利用计算机强有力的计算功能和高效率的图形处理能力,辅助知识劳动者进行工程和产品的设计与分析,以达到理想的目的或取得创新成果的一种技术。它是综合了计算机科学与工程设计方法的最新发展而形成的一门新兴学科。计算机辅助设计技术的发展是与计算机软件、硬件技术的发展和完善,与工程设计方法的革新紧密相关的。采用计算机辅助设计已是现代工程设计的迫切需要。CAD技术目前已广泛应用于国民经济的各个方面,其主要的应用领域有以下几个方面。
\subsubsection{制造业中的应用}

CAD技术已在制造业中广泛应用,其中以机床、汽车、飞机、船舶、航天器等制造业应用最为广泛、深入。众所周知,一个产品的设计过程要经过概念设计、详细设计、结构分析和优化、仿真模拟等几个主要阶段。

同时,现代设计技术将并行工程的概念引入到整个设计过程中,在设计阶段就对产品整个生命周期进行综合考虑。当前先进的CAD应用系统已经将设计、绘图、分析、仿真、加工等一系列功能集成于一个系统内。现在较常用的软件有UG II、I-DEAS、CATIA、PRO/E、Euclid等CAD应用系统,这些系统主要运行在图形工作站平台上。在PC平台上运行的CAD应用软件主要有Cimatron、

Solidwork、MDT、SolidEdge等。由于各种因素,目前在二维CAD系统中Autodesk公司的AutoCAD占据了相当的市场。

\subsubsection{工程设计中的应用}

CAD技术在工程领域中的应用有以下几个方面:

(1)建筑设计,包括方案设计、三维造型、建筑渲染图设计、平面布景、建筑构造设计、小区规划、日照分析、室内装潢等各类CAD应用软件。

(2)结构设计,包括有限元分析、结构平面设计、框/排架结构计算和分析、高层结构分析、地基及基础设计、钢结构设计与加工等。

(3)设备设计,包括水、电、暖各种设备及管道设计。

(4)城市规划、城市交通设计,如城市道路、高架、轻轨、地铁等市政工程设计。

(5)市政管线设计,如自来水、污水排放、煤气、电力、暖气、通信(包括电话、有线电视、数据通信等)各类市政管道线路设计。

(6)交通工程设计,如公路、桥梁、铁路、航空、机场、港口、码头等。

(7)水利工程设计,如大坝、水渠、河海工程等。

(8)其他工程设计和管理,如房地产开发及物业管理、工程概预算、施工过程控制与管理、旅游景点设计与布置、智能大厦设计等。
\subsubsection{电气和电子电路方面的应用}
CAD技术最早曾用于电路原理图和布线图的设计工作。目前,CAD技术已扩展到印刷电路板的设计(布线及元器件布局),并在集成电路、大规模集成电路和超大规模集成电路的设计制造中大显身手,并由此大大推动了微电子技术和计算及技术的发展。
\subsubsection{仿真模拟和动画制作}
应用CAD技术可以真实地模拟机械零件的加工处理过程、飞机起降、船舶进出港口、物体受力破坏分析、飞行训练环境、作战方针系统、事故现场重现等现象。用计算机从事艺术创作,计算机图形学除了广泛用于艺术品的制造,如各种图案、花纹及传统的油画、中国国画等。还成功的用来制造广告、动画片甚至电影,其中有的影片还获得了奥斯卡奖。这是电影界最高的殊荣。目前国内外不少人士正在研制人体模拟系统,这使得在不久的将来把历史上早已去世的著名影视明星重新搬上新的影视片成为可能。这是一个传统的艺术家无法实现也不可想象的。

\subsubsection{其他应用}
CAD技术除了在上述领域中的应用外,在轻工、纺织、家电、服装、制鞋、医疗和医药乃至体育方面都会用到CAD技术

CAD标准化体系进一步完善;系统智能化成为又一个技术热点;集成化成为CAD技术发展的一大趋势;科学计算可视化、虚拟设计、虚拟制造技术是20世纪90年代CAD技术发展的新趋向。

经过了一阶段计算机图形学的学习,对于图形学中基本图形的生成算法有了一定的了解。深度研究图形学,需要高深的数学知识,且每一个细化的方向需要的知识也不一样。图形学是计算机科学与技术学科的活跃前沿学科,被广泛的应用到

生物学、物理学、化学、天文学、地球物理学、材料科学等领域。我深深感到这门学科涉及的领域之广是惊人的,可以说博大精深。

\section{计算机图形学发展前景}

综观计算机图形学的发展,我们发现图形学的发展迅速,并且已经成为一门独立的学科,傲站在科学的前端。计算机图形学的已经应用到各个领域。比如计算机辅助设计与制造,自然景物仿真和计算机动画。在我们的生活到处可见,使我们的生活变的绚丽多彩。还有将可视化用于天气预报,使气象预报越来越准确;用于地质堪探,使地质学家可以发现新资源;用于医学做一些精密的手术提高了人们的寿命等。总之计算机图形学的应用给人类带来了很多益处,在促进人们物质水平提高的同时,也给我们带来的精神上的享受。当然,计算机图形学在某些领域的发展还未成熟,需要图形学工作者再接再厉,不断完善它的不足之处。从长远来看。计算机图形学有着广泛的发展前景,而且将在人们的生活中起着越来越重要的作用。



\end{document}

