\documentclass[UTF8]{article}
\usepackage{ctex}
\usepackage{listings}
\usepackage[bf,small,indentafter,pagestyles]{titlesec} 

\usepackage{geometry}  %边距和纸张大小
\geometry{a4paper}
\geometry{left=2cm,right=2cm,top=3cm,bottom=3cm}
\usepackage{fancyhdr}
\usepackage{url}
 %使用graphicx包
\usepackage{graphicx}
\usepackage{amsmath}
\usepackage{amssymb} %使用数学符号使用 
\usepackage{cite} %使用引用 
\usepackage{enumerate} %使用item使用

\begin{document} 
\author{    彭峰     05121214   }
 \date{2015-5-31}
\title{计算机图形学}  %题目要改掉   后面
 \titlespacing{\chapter}{0pt}{*0}{*4}
\titlelabel{\S\thetitle\quad}
\maketitle%这是无目录只有title的部分


\section{综述}
分别从常见的硬件设备,图像数字化   -----  等等方面进行阐述,旨在了解计算机图形学可能使用的主要知识点。
在这里,我们假设一些常见的名词是已知的,并不对其作出解释。
\section{硬件设备}
硬件设备主要是将自然界的图像转变为计算机能够识别的图像,这是一个输入输出转化的过程,在进行图像记录和输入时,一些硬件是需要的,同时,为了能够将数字图像以某种形式输出,我们需要一些输出设备。
图像数字化需要有下面五个部分:
采样孔:用于减小周围噪声,定点观测特定的图像元素,可以添加过滤装置。
图像扫描结构:设备的框架部分,用来良好的收集到图样。
光传感器:一般是将光强参数转变为电压或者电流的变换器。
量化器:函数表达方式通常用连续函数和离散函数,这里通常是将连续的物理量转变为离散的物理量,一般情况是模数转换电路。
存储:输出的大量数值存储到存储器中便于进行后期的处理和加工。
常见的设备有:扫描仪,摄像机,照相机等。


\subsection{扫描仪}
是一个能够把照片、印刷文件或手写文件等视频,或装饰品等小对象扫描、分析并化成数字视频的器材。通常可通过感光CCD 阅读红绿蓝3原色(RGB) 数据。存在时间较长,通常只在部分专业领域使用。普通的可以通过手机来代替。
\subsection{摄像机}
此处的图像传感器包括光电二极管阵列,电荷耦合器件,CMOS传感器,光电导成像四种的转换方式。
光电二极管通过包含一个光电二极管传感器阵列,一组开关和相关控制电路的方式,来实现光电转换。
电荷耦合器件:是一个基于模拟信号的设备。当光投射到其表面时,将有信号电荷产生。电荷信号可以被转换成电压,并按指定的时序将图像信息输出。数码相机主板上的其他电路将把这信号转换成数字信号,以便微处理器进行处理。
CMOS传感器器件:每个光电传感器附近都有相应的电路直接将光能量转换成电压信号。功耗较低。
光电导成像:其类似于CRT的扫描和成像方式。光学图像转换后通过电子打击局部放电的形式形成电子图像。
现主要使用的是CCD和CMOS传感器,其他已经接近淘汰。
\subsection{照相机}

\subsection{显示器}
主要的包括CRT显示器和液晶显示器,CRT显示器已经淘汰,液晶显示器经过多代的更新后已是市场主流,有多种不同的类型。包括blabla
\subsection{打印机}

\subsection{计算机}
\subsection{其他}
除此之外还有人们所熟悉的键盘,鼠标,还有图形输入板、扫描仪、光笔、游戏杆、跟踪球、触摸屏和语音系统等。

\section{图像数字化}




\section{常见API和平台}
\subsection{OpenGL}
开放图形库(英语:Open Graphics Library,缩写为OpenGL)是个定义了一个跨编程语言、跨平台的应用程序接口(API)的规范,它用于生成二维、三维图像。这个接口由近三百五十个不同的函数调用组成,用来从简单的图形比特绘制复杂的三维景象。而另一种程序接口系统是仅用于Microsoft Windows上的Direct3D。OpenGL常用于CAD、虚拟实境、科学可视化程序和电子游戏开发。

OpenGL的高效实现(利用了图形加速硬件)存在于Windows,很多UNIX平台和Mac OS。这些实现一般由显示设备厂商提供,而且非常依赖于该厂商提供的硬件。
\subsection{OpenCV}
OpenCV的全称是Open Source Computer Vision Library,是一个跨平台的计算机视觉库。OpenCV是由英特尔公司发起并参与开发,以BSD许可证授权发行,可以在商业和研究领域中免费使用。OpenCV可用于开发实时的图像处理、计算机视觉以及模式识别程序。OpenCV用C++语言编写,它的主要接口也是C++语言,但是依然保留了大量的C语言接口。该库也有大量的Python, Java and MATLAB/OCTAVE (版本2.5)的接口。这些语言的API接口函数可以通过在线文档获得。[4]现在也提供对于C\#, Ruby的支持。所有新的开发和算法都是用C++接口。一个使用CUDA的GPU接口也于2010年9月开始实现.
\subsection{DirectX}
DirectX(Direct eXtension,缩写:DX)是由微软公司创建的一系列专为多媒体以及游戏开发的应用程序接口。旗下包含了Direct3D、Direct2D、DirectCompute等等多个不同用途的子部分,因为这一系列API皆以Direct字样开头,所以DirectX(只要把X字母替换为任何一个特定API的名字)就成为了这一巨大的API系列的统称。目前最新版本为DirectX 12,随附于Windows 10操作系统之上。

\subsection{Qt}
Qt是一个跨平台的C++应用程序开发框架。广泛用于开发GUI程序,这种情况下又被称为部件工具箱。也可用于开发非GUI程序,比如控制台工具和服务器。Qt使用标准的C++和特殊的代码生成扩展(称为元对象编译器(Meta Object Compiler, moc))以及一些宏。
\subsection{GTK+}
GTK+最初是GIMP的专用开发库(GIMP Toolkit),后来发展为Unix-like系统下开发图形界面的应用程序的主流开发工具之一。GTK+是自由软件,并且是GNU计划的一部分。GTK+的许可协议是LGPL。GTK+使用C语言开发,但是其设计者使用面向对象技术。也提供了C++(gtkmm)、Perl、Ruby、Java和Python(PyGTK)绑定,其他的绑定有Ada、D、Haskell、PHP和所有的.NET编程语言。


%% \section{常用处理软件}  同上
\section{基本图形计算方法}
\subsection{点运算}
\subsection{代数运算}
\subsection{几何运算}
%%这里的有些问题,需要准备好



\section{图像处理}
图像复原
图像压缩

\section{模式识别}
%%这里看的自己会不会写,反正。。。




\end{document}

